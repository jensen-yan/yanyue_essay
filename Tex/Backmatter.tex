%---------------------------------------------------------------------------%
%->> Backmatter
%---------------------------------------------------------------------------%
\chapter[致谢]{致\quad 谢}\chaptermark{致\quad 谢}% syntax: \chapter[目录]{标题}\chaptermark{页眉}
%\thispagestyle{noheaderstyle}% 如果需要移除当前页的页眉
%\pagestyle{noheaderstyle}% 如果需要移除整章的页眉

在论文即将完成之际,回顾研究生三年的生涯,内心许多感慨,在此向所有关心、支持和帮助过我的人表示衷心的感谢!

我要感谢党和国家,我出生农村,家里经济条件受限,在初中阶段一直到本科毕业,
一直都获得过国家的贫困生资助,让我能够顺利完成学业。
感谢党和国家对基层教育和公平性的重视,让我有机会从农村,走到省会城市成都,再到国家首都北京,接受更好的教育。

我要感谢我的母校中国科学院大学,我在此完成了本科和硕士的学业。
本科依托于中国科学院的优质教育资源,
我在此奠定了扎实的数学、物理、计算机基础,
本科的操作系统课程、组成原理课程和体系结构课程都让我对计算机系统有了更深的理解,
促使我选择了计算机系统结构作为硕士的研究方向。
硕士期间在中国科学院计算技术研究所和龙芯中科技术有限公司联合成立的龙芯实验室下,
我有幸参与到二进制翻译系统的分析与优化工作,
我在此获得了很多的知识和经验,也认识了许多优秀的同学和老师。

我要向我的硕士生导师王剑老师表示最诚挚的感谢,王老师在我硕士期间给予了我很多的指导和帮助,
并在我研究方向的选择上给予了很多的建议,让我能够顺利完成硕士学业。
此外我还想感谢课题组的张福新老师,他作为龙芯实验室主任和二进制翻译领域的专家,
在我的研究工作中给予了我很多的建议和指导,让我受益匪浅。

我要向实验室师兄谢本壹博士表示感谢,谢博士在带领我入门二进制翻译领域的过程中给予了我很多的帮助,
和谢博士共同完成的两个课题,也让我在代码调试、实验分析制图和论文写作等方面都得到了很多的锻炼。
我还要感谢实验室的其他师兄师姐们、同学们、学弟学妹们,
特别是李欣宇师兄、燕澄皓同学、张壮壮同学、刘天义师兄、张婷婷师姐、陶思成师弟、吴翔师弟等,
我们共同完成的课题让我学到了很多,也感受到了团队合作的重要性。

我还要感谢我的女朋友袁嘉怡,感谢她在我本科和研究生期间对我的支持和鼓励,
让我多次走出困境,重新振作,她的陪伴让我感到很幸福!
她也让我在学业和生活上都能够做到更好的平衡!

最后我要感谢我的父母,是他们的无私付出和默默陪伴,让我能走到今天,他们是我永远的依靠和榜样!

忠心感谢所有关心、支持和帮助过我的人,谢谢你们!

\rightline{2024年6月}
\chapter{作者简历及攻读学位期间发表的学术论文与其他相关学术成果}

\section*{作者简历:}
晏悦,四川成都人,1998年出生。

2017年09月——2021年06月,在中国科学院大学计算机科学与技术学院获得学士学位。

2021年09月——2024年06月,在中国科学院计算技术研究所攻读硕士学位。

\section*{已发表(或正式接受)的学术论文:}

{
\setlist[enumerate]{}% restore default behavior
\begin{enumerate}[nosep]
    \item Xie, Benyi, Xinyu Li, \textbf{Yue Yan}, Chenghao Yan, Tianyi Liu, Tingting Zhang, Chao
    Yang and Fuxin Zhang. “On-Demand Triggered Memory Management Unit
    in Dynamic Binary Translator.”Advanced Parallel Programming Technologies
    (2023).

    \item Xie, Benyi, \textbf{Yue Yan}, Chenghao Yan, Sicheng Tao, Zhuangzhuang Zhang, Xinyu
    Li, Yanzhi Lan, Xiang Wu, Tianyi Liu, Tingting Zhang and Fuxin Zhang.
    “An Instruction Inflation Analyzing Framework for Dynamic Binary Translators.”ACM
    Transactions on Architecture and Code Optimization (2024).
    % \item 已发表的工作2
\end{enumerate}
}


\section*{参加的研究项目:}
\begin{enumerate}[nosep]
    \item 二进制翻译膨胀分析工作\ 合作参与者
    \item x86微译器项目\ 合作参与者
    \item RISC-V微译器项目\ 负责人
\end{enumerate}


\section*{获奖情况:}
\begin{enumerate}[nosep]
    \item 2023年\ 中国科学院计算技术研究所\ 所长优秀奖
    \item 2023年\ 中国科学院大学\ 三好学生
    \item 2023年\ 龙芯中科技术有限公司\ 龙芯实验室优秀学生
\end{enumerate}

\cleardoublepage[plain]% 让文档总是结束于偶数页,可根据需要设定页眉页脚样式,如 [noheaderstyle]
%---------------------------------------------------------------------------%
