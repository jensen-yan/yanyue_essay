%---------------------------------------------------------------------------%
%->> Backmatter
%---------------------------------------------------------------------------%
\chapter[致谢]{致\quad 谢}\chaptermark{致\quad 谢}% syntax: \chapter[目录]{标题}\chaptermark{页眉}
%\thispagestyle{noheaderstyle}% 如果需要移除当前页的页眉
%\pagestyle{noheaderstyle}% 如果需要移除整章的页眉

此处填写致谢。


\rightline{2023年6月}
\chapter{作者简历及攻读学位期间发表的学术论文与其他相关学术成果}

\section*{作者简历:}
晏悦,四川成都人,1998年出生。

2017年09月——2021年06月,在中国科学院大学计算机科学与技术学院获得学士学位。

2021年09月——2024年06月,在中国科学院计算技术研究所攻读硕士学位。

\section*{已发表(或正式接受)的学术论文:}

{
\setlist[enumerate]{}% restore default behavior
\begin{enumerate}[nosep]
    \item Xie, Benyi, Xinyu Li, \textbf{Yue Yan}, Chenghao Yan, Tianyi Liu, Tingting Zhang, Chao
    Yang and Fuxin Zhang. “On-Demand Triggered Memory Management Unit
    in Dynamic Binary Translator.”Advanced Parallel Programming Technologies
    (2023).

    \item Xie, Benyi, \textbf{Yue Yan}, Chenghao Yan, Sicheng Tao, Zhuangzhuang Zhang, Xinyu
    Li, Yanzhi Lan, Xiang Wu, Tianyi Liu, Tingting Zhang and Fuxin Zhang.
    “An Instruction Inflation Analyzing Framework for Dynamic Binary Translators.”ACM
    Transactions on Architecture and Code Optimization (2024).
    % \item 已发表的工作2
\end{enumerate}
}


\section*{参加的研究项目:}
\begin{enumerate}[nosep]
    \item 二进制翻译膨胀分析工作\ 合作参与者
    \item x86微译器项目\ 合作参与者
    \item RISC-V微译器项目\ 负责人
\end{enumerate}


\section*{获奖情况:}
\begin{enumerate}[nosep]
    \item 2023年\ 中国科学院计算技术研究所\ 所长优秀奖
    \item 2023年\ 中国科学院大学\ 三好学生
    \item 2023年\ 龙芯中科技术有限公司\ 龙芯实验室优秀学生
\end{enumerate}

\cleardoublepage[plain]% 让文档总是结束于偶数页,可根据需要设定页眉页脚样式,如 [noheaderstyle]
%---------------------------------------------------------------------------%
