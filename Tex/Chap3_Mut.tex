\chapter{微译器介绍}\label{chap:MUT}

\section{微译器架构细节}

\begin{figure}[h]
  \centering
  \includegraphics[width=1\linewidth]{./image/front_end_transutor.pdf}
  \caption{架构细节图。}
  \label{img:front_end_transutor}
\end{figure}

首先简要回顾一下微译器的架构细节,如图\ref{img:front_end_transutor}:

\subsection{硬件部分}

在硬件部分,引入了\textbf{翻译缓存}(Translation Cache),该缓存作为一级缓存负责存储预翻译的微码指令集,替代了原本的微码缓存。
翻译缓存的每行组织形式和微码缓存类似,都是前面部分存放微码指令(这里为融合微码指令,下一节详细介绍),
后面存放立即数(主要为了方便一条变长的X86指令,能更好的翻译成定长的融合微码指令),
微码指令和立即数相向生长,中间可能有空洞,这也是新产生的性能开销。

此外与传统微码架构不同,微码缓存数据是直接来源于CPU和指令缓存;
而翻译缓存通过二进制翻译透过内存层次(从内存加载到L3 Cache, 再到L2Cache, 最后到微码缓存)进行填充,
取代了传统的指令缓存和译码器的角色。
在传统X86架构下,取指部件会同时查询指令缓存和微码缓存;
而在微译器架构下,取指部件仅查询翻译缓存,硬件的译码器被软件的二进制翻译器取代。

\subsection{软件部分}

在软件部分,引入了静态和动态二进制翻译器。
程序首先通过静态二进制翻译器被翻译成微指令,并被写入预翻译文件,存储在硬盘中。
在客户程序执行阶段,预翻译文件被加载到内存中,程序计数器被设置为客户程序的入口。
取指部件从翻译缓存中取指,若翻译缓存或内存层次命中,则与从指令缓存取指类似,不断取指执行。
若翻译缓存和内存层次均未命中(例如存在自修改代码等),说明客户指令还未翻译,此时会调用动态二进制翻译器进行实时翻译。



\section{融合微码介绍}
本节详细介绍重要的一个概念——融合微码。“融合”代表它能融合了多种指令集的特征,
包括X86和RISCV,未来还可以支持ARM,MIPS等其他指令集,
或许叫统一微码也更好理解。但融合并非简单的把所有指令集简单拼接就好,这会造成指令集冗余,挤占有限的编码空间。

“微码”这个名字主要由于它在翻译缓存中的组织形式和传统的X86微码缓存组织形式类似,并且更像是不同指令集的更低级表现形式;
此外目前第一代的融合微码是在原有的Gem5 X86微码上扩充的,
所以便于理解仍然保留融合微码的名字。

但事实上,融合微码有很大特征很像一个“RISC指令集”。
由于我们的融合微码指令是需要存储在磁盘中的,并不像传统X86微码只是作为一个“暂时指令”存在于CPU运行期间,
所以融合微码指令需要像普通指令集一样进行\textbf{编码},
编译器(或者二进制翻译器)把指令\textbf{编码}为定长的融合微码指令,并存储于磁盘中。
CPU再从磁盘中加载融合微码指令,\textbf{解码}为CPU识别的信息,这一阶段也叫“译码阶段”。
因此从这个角度来说,加上指令定长、只能操作寄存器、指令间关系被简化等特征,融合微码很像一个普通的“RISC指令集”。
因此融合微码类似传统指令集概念,也是有编码空间的概念。