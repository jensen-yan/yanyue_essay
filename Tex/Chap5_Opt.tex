\chapter{微译器优化方案}\label{chap:Opt}

前文中提到了微译器的引入能消除间接跳转的开销并缩小指令集语义差异,从而提升二进制翻译器的性能。
然而,微译器的引入也会带来一些新的问题,产生新的性能瓶颈,本节将首先分析微译器的开销来源,然后
针对这些问题提出一些优化方案。




\section{微译器开销来源}
微译器的主要开销来源于磁盘、内存、缓存三个方面,前两者来源于预翻译文件中\textbf{重复存储}机制,后者来源于翻译缓存的\textbf{存储效率}问题。
重复存储机制是为了减少动态二进制翻译的开销,提高性能,但是会增加磁盘和内存的开销;
存储效率问题是由于翻译缓存行的固定长度,导致存储效率较低,缓存缺失率较高,进而影响性能。
本小节主要分析重复存储机制,下一小节将分析存储效率问题并给出优化方案。

预翻译文件的作用和可执行文件(Linux中为ELF格式文件)类似,是存储程序的二进制指令的文件格式,处理器会从预翻译文件中加载融合微码指令集,进而译码执行。
预翻译文件的数据段和ELF文件数据段相同,但是由于重复存储机制,代码段相对于ELF文件的代码段会膨胀64倍(64为翻译缓存行长度),接下来会举例介绍预翻译文件的代码段格式。

\begin{figure}[!htbp]
  \centering
  \includegraphics[width=1\linewidth]{./image/aot_format.pdf}
  \caption{预翻译文件格式,A、B、C、D为客户指令,a0、b0、c0、d0为翻译出的融合微码指令,由于重复存储,会存储a0-b0-c0-d0、b0-c0-d0、c0-d0、d0这
    四行融合微码指令,并填充对齐。}
  \label{img:aot_format}
\end{figure}

如图\ref{img:aot_format},展示了预翻译文件的代码段格式,为了简单起见,假设图中的指令缓存行大小为8字节,包含4条客户指令A、B、C、D,
起始地址分别为0x0、0x2、0x3、0x6,
生成的融合微码指令为a0、b0、c0、d0。
对应的预翻译文件会有8行,每行长度为64字节(相对于ELF文件膨胀了64倍)。
第0行(地址0x0)存储A指令开始的所有融合微码指令a0、b0、c0、d0;
第2行(地址0x80)存储B指令开始的所有融合微码指令b0、c0、d0;
第3行(地址0xC0)存储C指令开始的所有融合微码指令c0、d0;
第6行(地址0x180)存储D指令开始的所有融合微码指令d0。
其余的第1、4、5、7行都是空行,用于填充对齐。
能看出来,ELF 文件中代码地址和预翻译文件中代码地址是线性映射关系,只需要简单的地址映射就能找到对应的融合微码指令,地址关系如下:

\begin{equation}
Addr_{AOT} = Base +  Addr_{ELF} \times 64 
\end{equation}

再回顾一下,为何需要\textbf{重复存储}融合微码指令呢?这是因为可能有指令跳转到一行的中间位置(例如跳转到指令C),
这样就需要从预翻译文件中加载C指令开始的融合微码指令。
如果只存储A指令开始的融合微码指令,那么在跳转到C指令时,就需要重新翻译C指令开始的融合微码指令,这样会增加额外的开销,
这个开销包括调用动态二进制翻译器进行实时翻译、翻译缓存的填充等,需要数百拍才能完成,因此为了减少这个开销,需要重复存储融合微码指令。


\textbf{重复存储}办法本质上是一种用空间换时间的策略,通过增加预翻译文件的大小,减少了动态二进制翻译的开销,提高了性能。
虽然文件代码段膨胀了64倍,但目前主流的SPEC 2017程序的代码段大小在几十MB,这样的膨胀对于现代存储设备来说并不是很大的开销。
而对于内存来说,预翻译文件中有大量的空行,可以通过压缩算法进行透明压缩,减少内存占用,Linux中的zswap技术就是这样的一种技术。
对于多级缓存来说,只有被取到的缓存行才会被加载到缓存中,未被取到的缓存行不会被加载到缓存中,例如图\ref{img:aot_format}中可能只有第一行被取到,其他行不会被加载到缓存中。


\section{翻译缓存优化}

本文设计中,翻译缓存是和指令缓存同级的(或者说,是用于\textbf{替换}指令缓存的),用于存放融合微码指令。
如果翻译缓存大小和指令缓存大小相同,总行数相同,那么一行中存放的指令数量越多,缓存的总指令数量就越多,
缓存缺失率就会越低,性能就会越好,
因此翻译缓存的存储效率对性能影响较大。
为了提升翻译缓存的存储效率,本文提出了一些优化方案,包括微码缓存优化、可变长微码行优化、融合指令集压缩编码优化等。
成果表明,这些优化方案能显著提升翻译缓存的存储效率,从而提升二进制翻译器的性能。

如前文\ref{sec:tcache}所述,翻译缓存的每行组织形式和微码缓存类似,都是前面部分存放微码指令,后面存放立即数,微码指令和立即数相向生长,中间可能有\textbf{空洞}。
此外对于64字节长度的翻译缓存行,本文还设计了一个16字节长的元信息部分,用于存储一些额外的信息,如指令类型、指令长度等。
以定长的4字节指令举例,一行指令缓存可以存放16条指令,而一行翻译缓存最多存放12条指令,存储效率只有75\%。
更严峻的是,由于有3个结束条件的限制,实际存放的指令数量可能更少,存储效率更低,根据实验,
平均一行翻译缓存只能存放5条指令,存储效率只有31.25\%。

首先回顾下空洞产生的原因,是由于三个结束条件的限制,导致翻译缓存行中的微码指令提前结束,不能填满整个缓存行。
相对于\ref{sec:complex_isa}小节中提到的微码缓存的3个结束条件,
翻译缓存结束条件基本相同:1. 遇到指令缓存行的结尾;2. 遇到控制流指令;3. 翻译缓存行已满。


翻译缓存是从微码缓存设计演化而来的,因此翻译缓存的空洞问题也是从微码缓存继承而来的。
\cite{kotraImprovingUtilizationMicrooperation2020}
对传统微码缓存的空洞问题进行了分析,放松了前两个结束条件,能够提升微码缓存的存储效率,进而提升了12\%的整体性能。
本文也借鉴了这个思路,对翻译缓存进行了优化,放松了前两个结束条件,提升了翻译缓存的存储效率。

\subsection{放松指令缓存行结尾}

前文中提到,遇到指令缓存行的结尾是一个结束条件,这样才能保证指令缓存行和翻译缓存行是一对多的关系。
当遇到自修改代码等情况时,需要刷新一行指令缓存行,这样才能方便找到对应的翻译缓存行进行刷新和替换。
然而,可以适当放松这个结束条件,允许连续的两行指令缓存行的指令填充到同一行翻译缓存行中。

\begin{figure}[!htbp]
    \centering
    \includegraphics[width=1\linewidth]{./image/aot_relaxed_icache.pdf}
    \caption{放松指令缓存行结尾的翻译缓存行组织形式,允许连续的两行指令缓存行的指令填充到同一组翻译缓存行中。
    下一行的指令e0和指令f0可以填充到上一个指令缓存行对应的翻译缓存行中。
    }
    \label{img:aot_relaxed_icache}
  \end{figure}

如图\ref{img:aot_relaxed_icache},展示了放松指令缓存行结尾的翻译缓存行组织形式。相对于\ref{img:aot_format}中的形式,
能允许下一行指令缓存行的指令填充到同一组翻译缓存行中,例如指令E和指令F可以填充到上一个指令缓存行对应的翻译缓存行中。
假如有指令跳转到C指令开始的基本块,就能直接取出c0-d0-e0-f0这四条微码指令(原本由于D指令为行结尾的结束条件,这一行只能存c0-d0指令),
这样就能减少这一行的空洞,提升了存储效率。

而对于自修改代码的处理,由于一行微码缓存行中的指令最多对应两行连续的指令缓存行,
例如c0-d0-e0-f0这四条微码指令对应的指令缓存行为A-B-C-D和E-F,所以只需要刷新对应的两行指令缓存行即可。
自修改代码的处理在实际场景中并不常见,并且对性能影响较小,因此放松这个结束条件是可行的。

\subsection{放松条件跳转指令}

前文中提到,控制流指令可能改变程序原本指令执行流,将顺序执行的指令分割为一个个基本块,因此控制流指令
是一个结束条件,这样能够保证翻译缓存行中的微码指令是一个基本块的连续指令。
控制流指令分为条件跳转和无条件跳转,
对于无条件跳转,一定会跳转到另一个基本块,作为结束条件是合理的;
但是对于条件跳转,在翻译过程中并不确定这条指令是否会跳转,如果不跳转,那么这条指令后面的指令也是连续的,
属于同一个基本块的,可以填充到同一行翻译缓存行中,因此可以适当放松条件跳转指令的结束条件。
(对于传统的微码缓存,预测为跳转的控制流指令才是一个结束条件,这是由于硬件译码可以快速判断是否跳转,但是对于微译器,
软件预翻译过程的“译码”并不能判断是否跳转,因此需要放松这个结束条件。)

对于所有的条件跳转指令,都可以放松这个结束条件,将这些指令和后续指令都可以填充到同一行翻译缓存行中。
如图\ref{img:aot_relaxed_cond},展示了放松条件跳转指令的翻译缓存行组织形式。对于条件跳转beq指令,可以将后续的指令也填充到同一行翻译缓存行中。
虽然在运行时,beq指令可能会跳转,后续的指令不会执行,但是对于不跳转的情况,后续的指令会执行,这样就能减少这一行的空洞,提升了存储效率。


\begin{figure}[!htbp]
    \centering
    \includegraphics[width=0.6\linewidth]{./image/aot_relaxed_cond.pdf}
    \caption{放松条件跳转的翻译缓存行(忽略了元信息和立即数),对于条件跳转beq指令,后续还能放微码;对于无条件跳转jr指令,后续不能放微码。}
    \label{img:aot_relaxed_cond}
  \end{figure}


\section{可变长翻译缓存行优化}

前文中提到过,每一个翻译缓存行中所有指令对应于一个基本块,根据本文插装分析的结果,每个基本块平均长度为5.5条指令。
然而,由于翻译缓存行的大小是固定的,为64字节,这意味着即便放松了结束条件,一行中有效的指令大约只有5条,占据20字节,存储效率平均只有35\%左右。

为此,本文提出了可变长翻译缓存行的优化方案,即根据基本块的长度动态调整翻译缓存行的大小,从而提升存储效率。
如图\ref{img:tcache_32},展示了可变长翻译缓存行的组织形式,
本文实现了两种长度的翻译缓存行,分别为32字节和64字节,根据基本块的长度动态选择合适的翻译缓存行,对于长度小于等于6的基本块,选择32字节的翻译缓存行,
对于长度大于6的基本块,选择64字节的翻译缓存行。
同时在翻译缓存行的元信息中增加了一个字段,用于存储翻译缓存行的长度,这样就能在加载翻译缓存行时,根据这个字段动态选择合适的翻译缓存行并解析其中的指令。

\begin{figure}[!htbp]
    \centering
    \includegraphics[width=0.8\linewidth]{./image/tcache_32.pdf}
    \caption{可变长翻译缓存行组织形式,有32字节和64字节两种长度,根据基本块的长度动态选择合适的翻译缓存行。}
    \label{img:tcache_32}
  \end{figure}


\section{融合指令集压缩编码优化}

前文中提到,借鉴RISCV 压缩指令集的思想,融合微码指令集也可以进行压缩编码,从而在一行翻译缓存中存放更多的指令,提升存储效率。
回顾图\ref{img:TISA_encoder},展示了融合微码可以有2字节和4字节两种长度,分别为压缩指令和标准指令。

压缩指令分为2操作数、1操作数和0操作数三种类型,编码槽位很少,编码空间有限,因此需要对指令集进行精心设计,找出最常用的指令,进行压缩编码。
本文通过QEMU插装工具分析了SPEC 2017中x86和RISCV指令集的常用指令,发现这些指令占据了大部分的指令数,如表\ref{tab:spec_insts}所示。
根据x86和RISCV指令集的常用指令,本文设计了一套融合指令集的压缩编码。

附录\ref{app:compact_insts}中列出了本文设计的压缩指令集,包括2操作数、1操作数和0操作数的指令,这些指令都是对应于标准指令的压缩版本。

\begin{table}[]
  \centering
  \caption{SPEC 2017中x86和RISCV指令集的前15个常用指令}
  \label{tab:spec_insts}
  \begin{tabular}{l|ll|ll}
     & x86指令            & 比例     & RISCV指令             & 比例     \\ \hline
  1  & mov\_R8\_M8      & 5.64\% & c.add\_R\_R         & 5.49\% \\
  2  & movsd\_R16\_M8   & 4.28\% & fld\_R\_M           & 4.64\% \\
  3  & mulsd\_R16\_R16  & 4.22\% & c.addi\_R\_IMM      & 4.53\% \\
  4  & addsd\_R16\_R16  & 3.66\% & fmul.d\_R\_R\_R     & 3.87\% \\
  5  & jne\_IMM8        & 3.17\% & add\_R\_R\_R        & 3.61\% \\
  6  & mov\_R8\_R8      & 3.04\% & c.ldsp\_R\_IMM\_R   & 3.27\% \\
  7  & je\_IMM8         & 2.97\% & c.mv\_R\_R          & 3.22\% \\
  8  & add\_R8\_IMM8    & 2.81\% & c.fld\_R\_IMM\_R    & 3.13\% \\
  9  & movapd\_R16\_R16 & 2.69\% & c.sdsp\_R\_IMM\_R   & 2.94\% \\
  10 & mov\_R4\_M4      & 2.59\% & fmadd.d\_R\_R\_R\_R & 2.80\% \\
  11 & movupd\_R16\_M16 & 2.17\% & fsub.d\_R\_R\_R     & 2.65\% \\
  12 & cmp\_R8\_R8      & 2.01\% & c.ld\_R\_IMM\_R     & 2.56\% \\
  13 & mov\_R4\_R4      & 1.91\% & ld\_R\_M            & 2.52\% \\
  14 & lea\_R8\_M8      & 1.83\% & bne\_R\_R\_IMM      & 2.25\% \\
  15 & subsd\_R16\_R16  & 1.67\% & fadd.d\_R\_R\_R     & 2.22\% \\
  \end{tabular}
  \end{table}


